\documentclass{letter}
\usepackage[letterpaper,margin=1in]{geometry}
\usepackage{pdflscape}
\usepackage{xcolor}
\usepackage{varwidth}
\newcommand\crule[3][red]{\textcolor{#1}{\rule{#2}{#3}}}
\usepackage{graphicx}
\renewcommand{\ttdefault}{pcr}
%\usepackage{units}                                                                 
\newcommand{\nf}[2]{{\footnotesize \raisebox{2pt}{#1}\ensuremath{\mkern-2mu}\ensuremath{\mkern-1mu}\raisebox{-2pt}{#2}}}
\usepackage{array}
\usepackage{enumitem}

%%%%%%%%%%%%%%%%%%%%%%%%%%%%%%%%%%%%%%%%%%%%%%%%%%
% Notes - other interesting links
% http://www.heise.de/newsticker/meldung/Mit-kuenstlicher-Intelligenz-alte-Handschriften-entziffern-2432125.html?wt_mc=rss.ho.beitrag.atom
% http://openstax.org/
% http://cnx.org/
% http://aimath.org/textbooks/
% http://www.perseus.tufts.edu/hopper/text?doc=Perseus%3atext%3a1999.01.0134
%%%%%%%%%%%%%%%%%%%%%%%%%%%%%%%%%%%%%%%%%%%%%%%%%%

% Sloan - Letters of Inquiry

% Letters of inquiry regarding the possibility of Foundation support should include:

%    A brief statement (two or three sentences) about the nature and purpose of the proposed project;
%    A rough estimate of the cost of the proposed project and the amount of funds the proposer will be seeking from the Foundation;
%    A rough estimate of the duration of the proposed project;
%    Your title and contact information;
%    The names and affiliations of other key members of the project, if any.

% Letters of inquiry should be: 

%    No more than one full page;
%    Submitted by email to the program director of the Alfred P. Sloan Foundation grant program from which you wish to receive funds. Please consult the staff directory on the Alfred P. Sloan Foundation Web site for contact information for each program director. If a program has more than one program director listed, please send the letter of inquiry to only one of the program directors.

% While we try to respond to all letters of inquiry quickly, this is not always feasible due to the large volume of inquries we receive and the time constraints faced by Alfred P. Sloan Foundation staff. If more than a month has passed since you submitted your letter of inquiry, it is appropriate to email the program director to whom you submitted your letter of inquiry and inquire about its status.

\begin{document}
\signature{Volker Sorge\\ \emph{for the GDML Consortium}}
\address{Volker Sorge\\
School of Computer Science\\
University of Birmingham\\
B15 2TT, UK}
\begin{letter}{Grants Department \\ Sloan Foundation}
\opening{Dear Sir or Madam:}

%    A brief statement (two or three sentences) about the nature and purpose of the proposed project;
%    A rough estimate of the cost of the proposed project and the amount of funds the proposer will be seeking from the Foundation;
%    A rough estimate of the duration of the proposed project;
%    Your title and contact information;
%    The names and affiliations of other key members of the project, if any.

This proposal has been drafted by a group of interested persons in
response to the report \emph{Developing a 21st Century Global Library
  for Mathematics Research} written under the auspices of the National
Research Council, and given further form as a new Global Digital
Mathematics Library initiative.

We would like to build a public mathematics library by bringing
together the world's free mathematics material in one place, using
existing open source software tools.
%
We will populate the library with mathematical material, including
books in the public domain, and research articles available under a
non-restrictive copyright license.
%
We will pilot features that facilitate computer-mediated and
computationally-driven improvements to the library's contents, to
develop further in collaboration with the recently-convened Global
Digital Mathematics Library working group.

%% User studies: will people use the library? - if we go with what
%% other people already have found and produced.  How do we get the
%% content and material to work together?  Would the average
%% scientist, for example, use this.

The expected full cost of the project is \$146,000, including
personnel and material costs.  Project members will be independent
contractors, which will keep overheads at a minimum, but note that we
would request that the Sloan Foundation fund this entire amount.  We
are setting up an organizational structure through the American
Mathematical Society, which has generously offered to donate time,
infrastructure, and an extensive corpus of public domain content.

We expect the project to run 18 months, during which we would expend
around 6,750 person hours.

The primary participants in the project are:

\begin{itemize}[label={},itemsep=-5pt]
\item {\scshape Volker Sorge} (University of Birmingham, UK)\\
\item {\scshape Peter Krautzberger} (American Mathematical Society, USA)\\ 
\item {\scshape Raymond Puzio} (Albert Einstein College of Medicine, USA) \\
\item {\scshape Joseph Corneli} (Goldsmiths College, University of London, UK) \\
\item {\scshape Mitar Milutinovic} (PeerLibrary \& UC Berkeley, USA) \\
\item {\scshape Michael Furlough} (Hathi Trust, USA) \\
\item {\scshape Dan Cohen} (Digital Public Library of America, USA)
\end{itemize}

We hope that we have inspired interest in a full description of the
proposed project.

\closing{With regards,}
\end{letter}

% \newpage
% \thispagestyle{empty}


\end{document}
