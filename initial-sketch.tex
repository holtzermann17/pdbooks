\documentclass{article}
\usepackage[letterpaper,margin=1in]{geometry}
\usepackage{pdflscape}
\usepackage{xcolor}
\usepackage{varwidth}
\newcommand\crule[3][red]{\textcolor{#1}{\rule{#2}{#3}}}
\usepackage{graphicx}
\renewcommand{\ttdefault}{pcr}
%\usepackage{units}                                                                 
\newcommand{\nf}[2]{{\footnotesize \raisebox{2pt}{#1}\ensuremath{\mkern-2mu}\ensuremath{\mkern-1mu}\raisebox{-2pt}{#2}}}
\usepackage{array}
\usepackage{enumitem}

%%%%%%%%%%%%%%%%%%%%%%%%%%%%%%%%%%%%%%%%%%%%%%%%%%
% Notes - other interesting links
% http://www.heise.de/newsticker/meldung/Mit-kuenstlicher-Intelligenz-alte-Handschriften-entziffern-2432125.html?wt_mc=rss.ho.beitrag.atom
% http://openstax.org/
% http://cnx.org/
% http://aimath.org/textbooks/
% http://www.perseus.tufts.edu/hopper/text?doc=Perseus%3atext%3a1999.01.0134
%%%%%%%%%%%%%%%%%%%%%%%%%%%%%%%%%%%%%%%%%%%%%%%%%%

% Sloan - Letters of Inquiry

% Letters of inquiry regarding the possibility of Foundation support should include:

%    A brief statement (two or three sentences) about the nature and purpose of the proposed project;
%    A rough estimate of the cost of the proposed project and the amount of funds the proposer will be seeking from the Foundation;
%    A rough estimate of the duration of the proposed project;
%    Your title and contact information;
%    The names and affiliations of other key members of the project, if any.

% Letters of inquiry should be: 

%    No more than one full page;
%    Submitted by email to the program director of the Alfred P. Sloan Foundation grant program from which you wish to receive funds. Please consult the staff directory on the Alfred P. Sloan Foundation Web site for contact information for each program director. If a program has more than one program director listed, please send the letter of inquiry to only one of the program directors.

% While we try to respond to all letters of inquiry quickly, this is not always feasible due to the large volume of inquries we receive and the time constraints faced by Alfred P. Sloan Foundation staff. If more than a month has passed since you submitted your letter of inquiry, it is appropriate to email the program director to whom you submitted your letter of inquiry and inquire about its status.



\begin{document}

\section*{Public Domain Books Proposal}

\subsection*{Phase 1. Library / Basic Platform}

We need to build a basic library platform that can be used throughout
the project, and that will hopefully be reused and extended by other
GDML participants in subsequent or parallel projects.  Likely
components include the existing open source tools that have been
developed by members of our consortium: BibServer (Python),
PeerLibrary (Javascript+HTML5), and Planetary (Drupal).  Although the
analogy is inexact these comprise approximate equivalents to a
physical library's \emph{catalog}, \emph{stacks}, and
\emph{collaborative work space}.
%
During the first phase of the project (Months 1-5) we will focus on
getting these software components interoperating and setting up
suitable public and password-protected deployments.

\subsection*{Phase 2. Content}

\begin{enumerate}[label=(\Alph*),itemsep=-5pt]
\item Populate the library with metadata about public domain math books.
\item Populate the library with links to scans where available.
\item Populate the library with ``born free'' works from sites like arXiv.
\item OCR using Infty for circa 5000 books.
\end{enumerate}

During the second phase of the project (Months 6-18) we will populate
the library with relevant data, relying primarily on our network of
library partners.  During this time we will continue to improve basic
interaction features of the library platform, and begin to work on
automated content tagging and metadata extraction.  We will deliver a
working prototype library showing some considerable improvements over
a digital raw archive.  Most current project go into infrastructure w/o content.  The proposal focuses on public domain
works, however we include a track devoted to works that are \emph{born
  digital} and that come with non-restrictive licensing terms.  We will take the small percentage of openly-licensed works on arXiv, and create a proof of concept showing the advantages of open licensing.  Importantly, we will emphasize the role of users in improving the material in the collection in various ways: with direct peer reviewed edits, with questions and comments, and interlinking with other related material.
%
There is quite a bit of content in a primative state of digitization: e.g. around 375000 pages of journals OCR'ed by the AMS that include the main text but no equations.  Google Books (goes from 1681 to 1921, and there is other available material from earlier WDML/EuDML efforts.  LaTeX is a good ``first step'' but we want to go further, with semantic markup that makes the content computationally accessible.  For purposes of generating high-quality LaTeX, OCR can be trained in (on long journal runs) with machine learning techniques.  Individual books present more challenges that invite data mining and quality control measures over the entire corpus.  Here, we can also draw computationally on arXiv and other digital but non-free sources.

\subsection*{Phase 3. Public facing platform with an eye to scaling up}

\begin{enumerate}[label=(\Alph*),itemsep=-5pt]
\item Wiki + MathJax/MathML for crowdsourcing improvements to the TeX generated by Infty
\item User study to define incentive structures suitable for motivating improvments to 5K-10K books
\item Improved pre- and post-processing for OCR, tools to assist proofreading process
\item Nice demo books showing what the digitized format has to offer
\end{enumerate}

During the third phase of the project (Months 19-30) we will focus on
features that facilitate ``crowdsourced'' improvements to the
library's contents.  We expect that a combination of volunteer and
paid effort will be most suited to digitizing and improving works.
Building suitable demos and usable public workflows will be necessary
to resolve the ``chicken and egg'' problem that would otherwise
present itself (i.e. no one is likely to contribute unless there is a
suitable platform with already-existing content of interest).  Learn from projects like Zooniverse.
%
Deliverables include a custom proof-reading interface built into Peer
Library, and integration with other open source tools like ShareLaTeX
that facilitate real-time interaction between users.  We will work with other GDML contributors to 
explore the requirements around integration with computational and formal languages.

Volker Sorge's research group is currently working on the semantic enrichment of
mathematical formulas by exploiting information extracted from the context of a
formula in a given document. This work can be directly integrated into the
improvment efforts to further enhance recognised content by not only providing
better semantic markup for formulas but also by enabling the interlinking of
related concepts inside documents as well as to resources elsewhere.



\newpage
\thispagestyle{empty}
\begin{landscape}

{\centering
{\ttfamily
{\Large \emph{Public Domain Books Project}}\\[.2cm]
{\large \emph{indicative Gantt chart}}\\[.2cm]
{\large \emph{September 18,2014}}\\[1cm]
\scalebox{1.2}{
\begin{tabular}{p{0.07\textwidth}p{0.07\textwidth}l@{\hspace{-0.01cm}}l@{\hspace{-0.01cm}}l@{\hspace{-0.01cm}}l@{\hspace{-0.01cm}}l@{\hspace{-0.01cm}}l@{\hspace{-0.01cm}}l@{\hspace{-0.01cm}}l@{\hspace{-0.01cm}}l@{\hspace{-0.01cm}}l@{\hspace{-0.07cm}}l@{\hspace{-0.07cm}}l@{\hspace{-0.07cm}}l@{\hspace{-0.07cm}}l@{\hspace{-0.07cm}}l@{\hspace{-0.07cm}}l@{\hspace{-0.07cm}}l@{\hspace{-0.07cm}}l@{\hspace{-0.07cm}}l@{\hspace{-0.07cm}}l@{\hspace{-0.07cm}}l@{\hspace{-0.07cm}}l@{\hspace{-0.07cm}}l@{\hspace{-0.07cm}}l@{\hspace{-0.07cm}}l@{\hspace{-0.07cm}}l@{\hspace{-0.07cm}}l@{\hspace{-0.07cm}}l@{\hspace{-0.07cm}}l@{\hspace{-0.07cm}}l@{\hspace{.07cm}}p{0.8\textwidth}}
\textbf{\emph{Phase}}&&        \textbf{1}&&&&&\textbf{2}&&&&&&&&&&&&&\textbf{3}&&&&&&\\
\textbf{\emph{Month}}&&        1&2&3&4&5&6&7&8&9&     \nf{1}{0}&\nf{1}{1}&\nf{1}{2}&\nf{1}{3}&\nf{1}{4}&\nf{1}{5}&\nf{1}{6}&\nf{1}{7}&\nf{1}{8}&\nf{1}{9}&\nf{2}{0}&\nf{2}{1}&\nf{2}{2}&\nf{2}{3}&\nf{2}{4}&\nf{2}{5}&\nf{2}{6}&\nf{2}{7}&\nf{2}{8}&\nf{2}{9}&\nf{3}{0}&\\[.7em]
30@0.25&Joe&    J&J&J&J&J&J&J&J&J&J&J&J&J&J&J&J&J&J&J&J&J&J&J&J&J&J&J&J&J&J& Basic platform; crowdsourcing, data mining\\
24@0.5& Mitar&  M&M&M&M&M&M&M&M&M&M&M&M&M&M&M&M&M&M& & & & & & &M&M&M&M&M&M& Peer Library, proofreading support; wiki\\
12@0.5& Ray&    R&R&R&R&R&R& & & & & & & & & & & & &R&R&R&R&R&R& & & & & & & Metadata workflow; OCR enhancements\\
24@0.5& Volker&  & & & & & &V&V&V&V&V&V&V&V&V&V&V&V&V&V&V&V&V&V&V&V&V&V&V&V& content extraction, enhancements\\
36@1.0&  S-Mit&  2&2&2&2&2&2&2&2&2&2&2&2&2&2&2&2&2&2& & & & & & & & & & & & & Peer Library\\
~6@1.0&  S-Ray&  1&1&1&1&1&1& & & & & & & & & & & & & & & & & & & & & & & & & Metadata wrangling\\
30@0.5& S-DPL&   & & & & & &5&5&5&5&5&5& & & & & & & & & & & & & & & & & & & Scanning\\
12@1.0&  S-Vo1&   & & & & & &1&1&1&1&1&1&1&1&1&1&1&1& & & & & & & & & & & & & OCR\\
12@1.0&  S-Vo2&   & & & & & & & & & & & & & & & & & &1&1&1&1&1&1&1&1&1&1&1&1& Content you can compute with\\
18@1.0&  S-Joe&  1&1&1&1&1&1&1&1&1&1&1&1&1&1&1&1&1&1& & & & & & & & & & & & & Interaction\\
\\
{\tiny months}
\fbox{114.5}
{\tiny hours} %
\fbox{18.3K}&&
\crule[blue!93]{.7em}{3.25em}& 
&
&
&
&
&
\crule[blue]{.7em}{3.5em}&
&
&
&
&
&
\crule[blue!85]{.7em}{3.0em}& 
&
&
&
&
&
\crule[blue!71]{.7em}{2.5em}& 
&
&
&
&
&
\crule[blue!64]{.7em}{2.25em}&  
&
&
&
&
&
\\
\end{tabular}}

\begin{flushright}
\vspace{-1.3in}
\begin{minipage}[t]{.65\textwidth}
{\footnotesize
Phase 1. Library / Basic Platform

Phase 2. Content
\begin{itemize}
\item  Populating the library with metadata about public domain math books
\item  Populating the library with links to scans where available
\item  Populating the library with ``born free'' works
\item  OCR using Infty for circa 5000 books
\end{itemize}
Phase 3. Public facing platform with an eye to scaling up 
\begin{itemize}
\item Wiki + MathJax/MathML for crowdsourcing improvements to the TeX generated by Infty
\item User study to define incentive structures suitable for motivating improvments to 5K-10K books
\item Improved pre- and post-processing for OCR, tools to assist proofreading process
\item Nice demo books showing what the digitized format has to offer
\end{itemize}
}
\end{minipage}
\end{flushright}
}
}
\end{landscape}

\end{document}
