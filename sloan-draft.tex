\documentclass[10pt,letterpaper]{article}
\usepackage[letterpaper,margin=1in]{geometry}
\usepackage{hyperref}
\usepackage{enumitem}
\setitemize{itemsep=-2pt}
\setenumerate{itemsep=-2pt,label={\Alph*)}}
\def\comment#1#2{\typeout{Comment! - #1: #2}}
\usepackage{xfrac}

\usepackage{soul}

\begin{document}
\thispagestyle{empty}

\noindent\textbf{Re: \quad Acquisition of Mathematical Content from Books and Journals in the Public Domain.}

\bigskip

\noindent Dear Dr.~Weber:

\bigskip

In response to the report \emph{Developing a 21st Century Global
  Library for Mathematics Research} and the corresponding Global
Digital Mathematics Library initiative we wish to propose a project
for funding by Sloan.  \textbf{In essence this proposal addresses the
  question: ``How can we put the mathematics public domain online?''}
The project will fill a critical gap in the state of retro-digitised
mathematical literature wherein mathematical formulas are
intentionally omitted, and thereby inaccessible to researchers and
computational software.  We propose a step-by-step process centred on
crowdsourcing corrections to errors in state-of-the-art mathematical
OCR.  \textbf{We will build a workflow for the comprehensive
  recognition, correction and curation of digitised mathematical
  literature, leveraging existing infrastructure:}
\begin{enumerate}
\item Identify textbooks, scholarly monographs, and journals that are
  in the public domain, and that may have suitable preliminary
  digitizations (i.e., scanned with basic text OCR).
\item Develop a robust process for an initial formula recognition from scanned 
documents, by employing and, where relevant, extending, existing OCR systems 
specialising in mathematics, such as {\sf Infty}.
\item Enable 
the manual correction of recognised mathematical formulae via crowdsourcing 
technology like Zooniverse's {\sf Scribe}.
\item Investigate and implement further enhancements leveraging
  experience from the EuDML project, e.g., interlinking formulas,
  definitions, and other key terms within and between documents and
  external sources (arXiv, Wikipedia, Mathworld, Peer Library,
  MathOverflow, PlanetMath, etc.) using the {\sf NNexus} autolinker
  developed by PlanetMath.
\end{enumerate}
\textbf{We envision this workflow serving as an engine for the
  creation of high-quality, fully accessible mathematical content in
  the future global digital mathematics library.}  We aim to produce a
comprehensive record of the mathematics literature in the public
domain, starting with metadata and scans and creating complete
digitizations following user demand.  We estimate there to be around
$10000$ mathematics books available under US copyright law.  Comparing
this to the $\approx$30000 \emph{articles} on mathematics in
Wikipedia, one sees the sense of scale and potential impact of the
project.  That is, even just considering books, the mathematical
public domain is around 100 times the scale of the mathematical
reference resources currently available on Wikipedia.  This project
will make these rich foundations of the world's mathematical heritage
accessible to anyone with an internet connection.

We estimate that the core effort will require 2\sfrac{1}{4} FTEs
working over 2\sfrac{1}{2} calendar years.  We plan to split this time
across five developers and include support for project management,
technical assistance with acquisitions and processing, and
dissemination of the results.  \textbf{The total costs are estimated
  to be \$544,000, over three phases: (I.~Library infrastructure)
  \$148,000; (II.~Content) \$248,000; (III.~Community features)
  \$148,000.}

As the first large-scale mathematical ``citizen science'' project we
believe this project can nurture a community for future digitization
and enhancement activities.  The major risk associated with the
project is community involvement to determine priorities for
digitization, and to deliver workpower for corrections and
enhancements.  Our strategy addresses this risk with \emph{(i)} robust
assistive technology, which earlier digitization efforts have not
offered; \emph{(ii)} attractive free content found nowhere else on the
web at present; and \emph{(iii)} working interconnections with
disparate contemporary mathematical crowdsourcing projects.


\bigskip

\begin{tabular}{p{.75\textwidth}r}
\textbf{Primary contact:} \par
\ul{Name and host institution TBD}
\par \emph{Contact information above.} &  \\[1cm]
\textbf{Key personnel:} &\\
\textsc{Volker Sorge} (University of Birmingham) &.25FTE \\
\textsc{Raymond Puzio} (Albert Einstein College of Medicine \& PlanetMath.org, Ltd.) & .25FTE \\
\textsc{Joseph Corneli} (Goldsmiths College, Univ. of London \& PlanetMath.org, Ltd.) &.25FTE \\
\textsc{Mitar Milutinovic} (PeerLibrary \& UC Berkeley)  &.25FTE \\
\textsc{Peter Krautzberger} (krautzource UG \& MathJax.org), \emph{project manager}&.25FTE\\
Name TBD, \emph{staff programmer}   &.5FTE \\
Name TBD, \emph{library technician}  &.5FTE \\
\end{tabular}

\end{document}

