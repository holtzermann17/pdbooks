\documentclass[10pt]{article}
\usepackage[letterpaper,margin=1in]{geometry}
\usepackage{hyperref}
\usepackage{enumitem}
\setitemize{itemsep=-2pt}
\setenumerate{itemsep=-2pt,label={\Alph*)}}
\def\comment#1#2{\typeout{Comment! - #1: #2}}
%%\footnote{Comment by #1: #2}}

%\usepackage{soul}

\begin{document}
\thispagestyle{empty}
%% Acquisition of Mathematical Content from Textbooks in the Public Domain.

Dear Dr.~Weber:

We\comment{VS}{AMS?} wish to propose a project for funding by Sloan in
response to the report \emph{Developing a 21st Century Global Library
  for Mathematics Research} written under the auspices of the National
Research Council, and given further form as a new Global Digital
Mathematics Library initiative.\comment{VS}{What's the relationship
  between Sloan and this?} The project will aim to bridge the gap
between the current state of retro-digitised mathematical literature,
where only textual content is recognised while mathematical formulas
are omitted, and the need to make the mathematical material fully
accessible to researchers and computational software
(e.g. screenreaders and search tools).

\textbf{We aim to leverage existing infrastructure to build a
  prototypical workflow for the comprehensive recognition, correction
  and curation of digitised mathematical literature.}

In particular we want to pursue the following goals:
\begin{enumerate}
\item Fully digitise the mathematical content of several scholarly
  monographs and journals that are in the public domain and that have
  already preliminary digitisations (i.e., scanned and processed with
  basic text recognition software, but without equations).
\item Develop a robust process for an initial mathematical formula
  recognition from scanned documents, by employing and, where
  relevant, extending, existing OCR systems specialising in
  mathematics, such as {\sf Infty}.
\item Develop a unified catalogue of public domain mathematics
  materials, and enable the manual correction of recognised
  mathematical formulae via crowdsourcing techniques, adapting a
  crowdsourcing platform like Zooniverse's {\sf Scribe}.
\item Building on techniques developed in the EuDML project,
  investigate and implement methods for further enhancement of
  content, e.g. via interlinking formulas and definitions, within and
  between documents hosted on PlanetMath and PeerLibrary as well as to
  outside sources like Arxiv, Wikipedia, Mathworld, etc., in support
  of effective curation and online provision to readers.
\end{enumerate}

We envision that the resulting mathematical content will not only be
directly useful for practitioners and learners, but also that the
inclusion of historical material will be of interest to historians and
other scholars and hobbyists, thereby nurturing a user community for
future correction and enhancement activities.  As such we envision the
proposed workflow to become a core component of a future global
digital mathematics library with a emphasis on making high-quality
mathematical content accessible by electronic means.

We estimate the project will require 5 developer years spread over 2
calendar years. We plan to split this time across four developers and
include support for the project manager and dissemination of the
results.  \textbf{The total costs are estimated to be \$544,000, over
  three phases:
(I.~Library infrastructure) \$148,000;
(II.~Content)               \$248,000;
(III.~Community features)   \$148,000.}

The primary contact for the group is \textsc{Peter Krautzberger}
(American Mathematical Society, USA).  Other key members include
\textsc{Volker Sorge} (University of Birmingham, UK), \textsc{Raymond
  Puzio} (Albert Einstein College of Medicine \& PlanetMath.org, Ltd.,
USA), \textsc{Joseph Corneli} (Goldsmiths College, University of
London, UK \& PlanetMath.org, Ltd., USA), and \textsc{Mitar
  Milutinovic} (PeerLibrary \& UC Berkeley, USA).

In addition, we are supported by \textsc{Thomas Fischer} (Universiut{\"a}tsbibliothek
G{\"o}ttingen, Germany), \textsc{Michael Furlough} (Hathi Trust, USA), \textsc{Dan Cohen} (Digital Public Library of America, USA), \textsc{Robert Simpson} (Zooniverse, UK).
\begin{flushright}
Sincerely,\\
\quad \\
\quad \\
\quad 
\end{flushright}
\end{document}

%%% Local Variables: 
%%% mode: latex
%%% TeX-master: t
%%% End: 
