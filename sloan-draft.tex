\documentclass[10pt]{article}
\usepackage[letterpaper,margin=1in]{geometry}
\usepackage{hyperref}
\usepackage{enumitem}
\setitemize{itemsep=-2pt}
\setenumerate{itemsep=-2pt,label={\Alph*)}}
\def\comment#1#2{\typeout{Comment! - #1: #2}}
\usepackage{xfrac}

\begin{document}
\thispagestyle{empty}

\noindent\textbf{Re: \quad Acquisition of Mathematical Content from Books and Journals in the Public Domain.}

\bigskip

\noindent Dear Dr.~Weber:

\bigskip

In response to the report \emph{Developing a 21st Century Global Library
for Mathematics Research} and the corresponding Global Digital
Mathematics Library initiative we wish to propose a project 
for funding by Sloan.

The project aims to fill a critical gap
in the state of retro-digitised mathematical literature
wherein mathematical formulas are intentionally omitted, 
and thereby inaccessible to researchers and computational software.
\textbf{We aim to build a prototypical workflow for the comprehensive 
recognition, correction
and curation of digitised mathematical literature in the public domain 
by leveraging existing infrastructure.} In particular, we want to pursue 
the following goals:
\begin{enumerate}
\item Identify scholarly monographs and journals that are in the
  public domain, and that may have a suitable preliminary
  digitizations (i.e., scanned with basic text OCR).
\item Develop a robust process for an initial formula recognition from scanned 
documents, by employing and, where relevant, extending, existing OCR systems 
specialising in mathematics, such as {\sf Infty}.
\item Enable 
the manual correction of recognised mathematical formulae via crowdsourcing 
technology like Zooniverse's {\sf Scribe}.
\item Investigate and implement further enhancements leveraging
experience from the EuDML project, e.g., 
interlinking formulas and definitions within and between documents and
external sources  (arXiv, Wikipedia, Mathworld, PlanetMath, Peer 
Library).
\end{enumerate}
\textbf{We envision the workflow outlined above serving as an engine
  for the creation of high-quality, fully accessible mathematical
  content in the future global digital mathematics library.}  The
resulting digitized mathematical content will be directly useful for
practitioners and learners as well as historians and other scholars.

We estimate that the project will require 2\sfrac{1}{4} FTEs working
over 2\sfrac{1}{2} calendar years.  We plan to split this time across
five developers and include support for project management, technical
assistance with acquisitions and processing, and dissemination of the
results.  \textbf{The total costs are estimated to be \$544,000, over
  three phases: (I.~Library infrastructure) \$148,000; (II.~Content)
  \$248,000; (III.~Community features) \$148,000.}

\bigskip

\begin{tabular}{p{.75\textwidth}r}
\textbf{Primary contact:} \par
Name and host institution TBD
\par \emph{Contact information above.} &  \\[1cm]
\textbf{Key personnel:} &\\
\textsc{Volker Sorge} (University of Birmingham) &.25FTE \\
\textsc{Raymond Puzio} (Albert Einstein College of Medicine \& PlanetMath.org, Ltd.) & .25FTE \\
\textsc{Joseph Corneli} (Goldsmiths College, Univ. of London \& PlanetMath.org, Ltd.) &.25FTE \\
\textsc{Mitar Milutinovic} (PeerLibrary \& UC Berkeley)  &.25FTE \\
\textsc{Peter Krautzberger} (krautzource UG \& MathJax), \emph{project manager}&.25FTE\\
Name TBD, \emph{staff programmer}   &.5FTE \\
Name TBD, \emph{library technician}  &.5FTE \\
\end{tabular}

\end{document}

