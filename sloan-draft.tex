\documentclass{article}
\usepackage[letterpaper,margin=1.5in]{geometry}
\usepackage{hyperref}
\usepackage{enumitem}
\setitemize{itemsep=-2pt}
\setenumerate{itemsep=-2pt,label={\alph*)}}
\def\comment#1#2{\typeout{Comment!}\footnote{Comment by #1: #2}}

\begin{document}
%% Acquisition of Mathematical Content from Textbooks in the Public Domain.

Dear Dr.~Weber,

We\comment{VS}{AMS?} wish to propose a project for funding by Sloan in response
to the report \emph{Developing a 21st Century Global Library for Mathematics
  Research} written under the auspices of the National Research Council, and
given further form as a new Global Digital Mathematics Library
initiative.\comment{VS}{What's the relationship between Sloan and this?} The
project will aim to bridge the gap between the current state of retro-digitised
mathematical literature, where only textual content is recognised while
mathematical formulas are omitted, and the need to transcribe the mathematical
content fully into electronic formats to make it amenable to search algorithms,
retrievable and reusable by researchers and computational software as well as
fully accessible by non-visual means.

We aim to leverage as much as possible existing infrastructure to build a
prototypical workflow for the comprehensive recognition, correction and curation
of digitised mathematical literature. In particular we want to pursue the
following goals:
\begin{enumerate}
\item Fully digitise the mathematical content of 3-4 mathematics books that are
  in the public domain and that have already beeen suitably digitised (i.e.,
  scanned at high resolution and process with text recognition). We will use
  material provided by the Universiut{\"a}tsbibliothek G{\"o}ttingen that will
  include also at least one historical book.
\item Develop a robust process for an intial mathematical formula recognition
  from scanned documents, by employing and possibly enhancing existing OCR
  systems specialising in Mathematics such as Infty.
\item Enable the manual correction of recognised mathematical formulae via
  crowdsourcing techniques. We aim adapt an crowdsourcing platform like
  Zooniverse.
\item Investigate and implement methods for further enhancement of content,
  e.g. via interlinking formulas, definition etc. within and between documents
  as well as to outside sources like Wikipedia, Mathworld, etc., its curation
  and its online provision to readers. We aim to adapt techniques developed in
  the context of the EuDML project while using PlanetMath as dedicated content
  delivery platform.
\end{enumerate}

We envision that the resulting mathematical content will not only be directly
useful for practitioners and learners, but also that in particular the inclusion
of historical material will be of interest for users might normally not be
considered the target audience of such an effort (e.g., retired Maths teachers),
thereby nurturing a user community available for similar correction activities
in the future.  As such we envision the proposed workflow to become a core
component of a future global digital mathematics library that has a clear
emphasis on making mathematical content amenable to electronic manipulation,
search and computational engines.

We estimate the project will require 5 developer years spread over 2 years. We
plan to split this time across four developers and include some additional
amount for the Project Manager and for dissemination of the results. A more
detailed breakdown is given below. The total costs are estimated to be
\$490,000\comment{VS}{This is pure fiction and probably too low!}.

\paragraph{Primary Contact}\label{primary-contact}

Peter? AMS?

\paragraph{Key members}\label{key-members}

\begin{itemize}
\item \textsc{Volker Sorge} (University of Birmingham, UK)
\item \textsc{Peter Krautzberger} (American Mathematical Society, USA)
\item \textsc{Raymond Puzio} (Albert Einstein College of Medicine, USA) 
\item \textsc{Joseph Corneli} (Goldsmiths College, University of London, UK)
\item \textsc{Mitar Milutinovic} (PeerLibrary \& UC Berkeley, USA) 
\end{itemize}

In addition, we are supported by \textsc{Thomas Fischer} (Universiut{\"a}tsbibliothek
G{\"o}ttingen), 
 \textsc{Michael Furlough} (Hathi Trust, USA),
 \textsc{Dan Cohen} (Digital Public Library of America, USA),
 \textsc{Robert Simpson} (Zooniverse, UK)

\end{document}

%%% Local Variables: 
%%% mode: latex
%%% TeX-master: t
%%% End: 
