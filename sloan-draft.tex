\documentclass[10pt]{article}
\usepackage[letterpaper,margin=1in]{geometry}
\usepackage{hyperref}
\usepackage{enumitem}
\setitemize{itemsep=-2pt}
\setenumerate{itemsep=-2pt,label={\Alph*)}}
\def\comment#1#2{\typeout{Comment! - #1: #2}}
%%\footnote{Comment by #1: #2}}

%\usepackage{soul}

\begin{document}
\thispagestyle{empty}
%% Acquisition of Mathematical Content from Textbooks in the Public Domain.

Dear Dr.~Weber:

In response to the report \emph{Developing a 21st Century Global Library
for Mathematics Research} and the renewed Global Digital
Mathematics Library initiative we wish to propose a project 
for funding by Sloan.

The project aims to fill a critical gap
in the state of retro-digitised mathematical literature
where mathematical formulas are intentionally omitted, preventing
mathematical material  to become fully
accessible to researchers and computational software.
\textbf{We aim to build a prototypical workflow for the comprehensive 
recognition, correction
and curation of digitised mathematical literature in the public domain 
by leveraging existing infrastructure.} In particular, we want to pursue 
the following goals:
\begin{enumerate}
\item Identify several scholarly monographs and journals that are in the public 
domain and have a suitable preliminary digitisations (i.e., scanned with 
text OCR). 
\item Develop a robust process for an initial formula recognition from scanned 
documents, by employing and, where relevant, extending, existing OCR systems 
specialising in mathematics, such as {\sf Infty}.
\item Develop a catalogue of public domain mathematics materials and enable 
the manual correction of recognised mathematical formulae via crowdsourcing 
technology like Zooniverse's {\sf Scribe}.
\item Investigate and implement further enhancements leveraging
experience from the EuDML project, e.g., by
interlinking formulas and definitions within and between documents as well as to 
external sources  (arXiv, Wikipedia, Mathworld, PlanetMath, Peer 
Library).
\end{enumerate}
We believe the resulting mathematical content will be
directly useful for practitioners and learners as well as historians and
other scholars. As the first mathematical ``citizen science'' project we 
believe this project can nurture a community for future digitization and 
enhancement activities. As such, we envision this workflow to become a 
core component for creating high-quality, fully accessible mathematical 
content in the future global digital mathematics library.


We estimate the project will require 5 developer years spread over 2
calendar years. We plan to split this time across four developers and
include support for the project manager and dissemination of the
results.  \textbf{The total costs are estimated to be \$544,000, over
  three phases:
(I.~Library infrastructure) \$148,000;
(II.~Content)               \$248,000;
(III.~Community features)   \$148,000.}

The primary contact for the group is \textsc{Peter Krautzberger}
(American Mathematical Society, USA).  Other key members include
\textsc{Volker Sorge} (University of Birmingham, UK), \textsc{Raymond
  Puzio} (Albert Einstein College of Medicine \& PlanetMath.org, Ltd.,
USA), \textsc{Joseph Corneli} (Goldsmiths College, University of
London, UK \& PlanetMath.org, Ltd., USA), and \textsc{Mitar
  Milutinovic} (PeerLibrary \& UC Berkeley, USA).

In addition, we are supported by \textsc{Thomas Fischer} (Universiut{\"a}tsbibliothek
G{\"o}ttingen, Germany), \textsc{Michael Furlough} (Hathi Trust, USA), \textsc{Dan Cohen} (Digital Public Library of America, USA), \textsc{Robert Simpson} (Zooniverse, UK).
\begin{flushright}
Sincerely,\\
\quad \\
\quad \\
\quad 
\end{flushright}
\end{document}

%%% Local Variables: 
%%% mode: latex
%%% TeX-master: t
%%% End: 
