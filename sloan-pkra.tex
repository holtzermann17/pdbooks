\documentclass[10pt,letterpaper]{article}
\usepackage[letterpaper,margin=1in]{geometry}
\usepackage{hyperref}
% \usepackage{enumitem}
% \setitemize{itemsep=-2pt}
% \setenumerate{itemsep=-2pt,label={\Alph*)}}
\def\comment#1#2{\typeout{Comment! - #1: #2}}
\usepackage{xfrac}

% \usepackage{soul}

\begin{document}
\thispagestyle{empty}

\noindent\textbf{Re: \quad A Citizen Science Project for Retro-Digitizing
Mathematics in the Public Domain.}

\bigskip

\noindent Dear Dr.~Weber:

\bigskip

In response to the creation of the Global Digital Mathematical 
Library Working Group (GDML WG) we wish to propose a 
project for funding by Sloan. 

We propose a citizen science project on the Zooniverse platform to 
crowdsource the identification of mathematical fragments in retro-digitized 
works as well as the verification of conversion and search of the resulting 
content. The results will provide the large scale, real world bodies of 
interconnected mathematical fragements and improve existing open-source 
mathematical OCR and search. We will develop a robust Zooniverse application 
where

\begin{enumerate}
\item users mark math fragements in digitized documents 
\item users verify and improve OCR results of these fragments,
\item users connect fragments across documents (e.g., 
identify variable names, labeled equations/theorems) 
\item users verify/improve search results. 
\end{enumerate}

To adapt to the crowd, we have identified a 
variety of openly-accessible, digitized works such as text books, monographs, 
personal archives of prominent mathematicians and other historical documents. 
The material will come from sources such as Google Books, 
Universit\"atsbibliothek G\"ottingen, and Project Euclid.

We believe this project will develop a vital component for making the world's 
mathematical heritage fully accessible to researchers and educators as well as 
computational tools.

We estimate that the core effort will require 2\sfrac{1}{4} FTEs
working over 2\sfrac{1}{2} calendar years.  We plan to split this time
across five developers and include support for project management,
technical assistance with acquisitions and processing, and
dissemination of the results.  \textbf{The total costs are estimated
  to be \$544,000, over three phases: (I.~Fragment identification)
  \$148,000; (II.~OCR and verification) \$248,000; (III.~connected fragments 
and search)
  \$148,000.}

As the first large-scale mathematical ``citizen science'' project we
believe this project can nurture a community for future digitization
and enhancement activities.  The major risk associated with the
project is community involvement to determine priorities for
digitization, and to deliver workpower for corrections and
enhancements.  Our strategy addresses this risk with \emph{(i)} robust
assistive technology, which earlier digitization efforts have not
offered; \emph{(ii)} attractive free content found nowhere else on the
web at present; and \emph{(iii)} working interconnections with
disparate contemporary mathematical crowdsourcing projects.


\bigskip

\begin{tabular}{p{.75\textwidth}r}
\textbf{Primary contact:} \par
\bf{Name and host institution TBD}
\par \emph{Contact information above.} &  \\[1cm]
\textbf{Key personnel:} &\\
\textsc{Volker Sorge} (University of Birmingham) &.25FTE \\
\textsc{Raymond Puzio} (Albert Einstein College of Medicine \& PlanetMath.org, 
Ltd.) & .25FTE \\
\textsc{Joseph Corneli} (Goldsmiths College, Univ. of London \& PlanetMath.org, 
Ltd.) &.25FTE \\
\textsc{Mitar Milutinovic} (PeerLibrary \& UC Berkeley)  &.25FTE \\
\textsc{Peter Krautzberger} (krautzource UG \& MathJax.org), \emph{project 
manager}&.25FTE\\
Name TBD, \emph{staff programmer}   &.5FTE \\
Name TBD, \emph{library technician}  &.5FTE \\
\end{tabular}


\end{document}
